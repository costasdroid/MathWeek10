\documentclass[greek]{beamer}
%\usepackage{fontspec}
%\newtheorem{definition}{Ορισμός}
\usepackage{amsmath,amsthm}
\usepackage{unicode-math}
\usepackage{xltxtra}
\usepackage{graphicx}
\usetheme{Warsaw}
\usecolortheme{seahorse}
\usepackage{hyperref}
\usepackage{ulem}
\usepackage{xgreek}
\usepackage{pgfpages}
%\setbeameroption{show notes on second screen}
%\setbeameroption{show only notes}

\setsansfont{Times New Roman}
\title[Διερεύνηση Θεμάτων]{Διερεύνηση Θεμάτων - Συμπληρωματικές Προτάσεις Μαθηματικών Προσανατολισμού Γ΄ λυκείου}
\author[Ελευθεριάδης, Λόλας, Ευαγγελόπουλος]{Μ. Ελευθεριάδης\inst{1} \and Κ. Λόλας\inst{2} \and Α. Ευαγγελόπουλος\inst{3}}
\institute[]
{
 \inst{1}%
 32ο ΓΕΛ ΘΕΣ/ΝΙΚΗΣ (ΠΕ03)
 \and
 \inst{2}%
 10ο ΓΕΛ ΘΕΣ/ΝΙΚΗΣ (ΠΕ03)
 \and
 \inst{3}%
 Σχ. Σύμβουλος Μαθηματικών
}
\date{Θεσσαλονίκη, Απρίλιος 2018}

\newtheorem{proposition}[theorem]{Πρόταση}

\begin{document}
\begin{frame}
 \titlepage
\end{frame}

\section{Εισαγωγή}
\begin{frame}{Γιατί}
 \begin{itemize}
  \item<1-> Δεν απαιτούνται
  \item<2-> Εμβαθύνουν στη θεωρία
  \item<3-> Βοηθούν στη κατανόηση
  \item<4-> Μας δίνεται η ευκαιρία να αποδείξουμε πόσο έξυπνοι είμαστε...
 \end{itemize}
\end{frame}

\section{Προτάσεις}
\begin{frame}{Πρόταση1}
 \begin{proposition}
  \normalfont Αν η συνάρτηση $f: \Delta \to R$ είναι γνησίως αύξουσα στο διάστημα Δ, τότε η εξίσωση $f( x ) = f^{-1} ( x )$ είναι ισοδύναμη με την εξίσωση $f( x ) = x$, $x\in \Delta $.
 \end{proposition}
\end{frame}

\begin{frame}{Πρόταση2}
 \begin{proposition}
  \normalfont Αν f είναι συνεχής και «1-1» στο διάστημα Δ, τότε η f είναι γνησίως μονότονη στο Δ.
 \end{proposition}
\end{frame}

\begin{frame}{Πρόταση3}
 \begin{proposition}
  \normalfont Δεν υπάρχει το όριο της $\varphi \left( x \right)=\eta \mu x$ στο $+\infty $.
 \end{proposition}
\end{frame}

\begin{frame}{Πρόταση4}
 \begin{proposition}
  \normalfont  Δεν υπάρχει το όριο της συνάρτησης φ(x) = ημx στο $\pm \,\infty $.
 \end{proposition}
\end{frame}

\begin{frame}{Πρόταση5}
 \begin{proposition}
  \normalfont Aν το $\underset{x\to {{x}_{0}}}{\mathop{\lim }}\,f(x)=0$ και $m\le g(x)\le M$ κοντά στο x0, τότε $\underset{x\to {{x}_{0}}}{\mathop{\lim }}\,[f(x)g(x)]=0$
 \end{proposition}
\end{frame}

\begin{frame}{Πρόταση6}
 \begin{proposition}
  \normalfont Αν η είναι περιττή συνάρτηση $f:R\to R$ και ισχύει $\underset{x\to {{0}^{+}}}{\mathop{\lim }}\,f(x)=+\infty $, τότε δεν υπάρχει το όριο της f στο 0.
 \end{proposition}


\end{frame}
\begin{frame}{Πρόταση7}
 \begin{proposition}
  \normalfont Aν η f είναι ( γνησίως) μονότονη στο Δ και υπάρχει το $\underset{x\to {{x}_{0}}}{\mathop{\lim }}\,f(x)$ στο $\mathbb{R}$, για κάθε $x_0 \in \Delta $, τότε η f είναι συνεχής στο Δ.
 \end{proposition}
\end{frame}

\begin{frame}{Πρόταση8}
 \begin{proposition}
  \normalfont Αν η f είναι συνεχής στο (α, β), $\underset{x\to \alpha }{\mathop{\lim }}\,f(x)=-\infty $ και $\underset{x\to \beta }{\mathop{\lim }}\,f(x)=+\infty $, τότε $f\left( (\alpha ,\beta ) \right)=\mathbb{R}$
 \end{proposition}
\end{frame}

\begin{frame}{Πρόταση9}
 \begin{proposition}
  \normalfont Αν φ είναι συνεχής και μη σταθερή στο [α, β] και φ(α)=φ(β), τότε η φ παρουσιάζει ακρότατο στο (α, β).
 \end{proposition}
\end{frame}

\begin{frame}{Πρόταση10}
 \begin{proposition}
  \normalfont Αν μία συνάρτηση f είναι συνεχής στο διάστημα (α, β) και «1-1», τότε η αντίστροφη της f είναι συνεχής.
 \end{proposition}
\end{frame}

\begin{frame}{Πρόταση11}
 \begin{proposition}
  \normalfont Αν f είναι συνάρτηση «1-1», παραγωγίσιμη στο ${{x}_{0}}$, ${{x}_{0}}\in \Delta $- διάστημα και ${f}'({{x}_{0}})\ne 0$ , τότε η ${{f}^{-1}}$είναι παραγωγίσιμη στο ${{y}_{0}}=f({{x}_{0}})$.
 \end{proposition}
\end{frame}

\begin{frame}{Πρόταση12}
 \begin{proposition}
  \normalfont Αν η συνάρτηση f είναι παραγωγίσιμη σε περιοχή του ${{x}_{0}}$, στο ${{x}_{0}}$ και υπάρχει το $\underset{x\to {{x}_{0}}}{\mathop{\lim }}\,{f}'(x)$, τότε η ${{f}^{'}}$ είναι συνεχής στο ${{x}_{0}}$
 \end{proposition}
\end{frame}

\begin{frame}{Πρόταση13}
 \begin{proposition}
  \normalfont Αν η $f$είναι συνεχής στο ${{x}_{0}}$, $f({{x}_{0}})\ne 0$ και η ${{f}^{2}}$είναι παραγωγίσιμη στο ${{x}_{0}}$ τότε η $f$ είναι παραγωγίσιμη στο ${{x}_{0}}$.
 \end{proposition}
\end{frame}

\begin{frame}{Πρόταση14}
 \begin{proposition}
  \normalfont Αν η συνάρτηση $f$ είναι παραγωγίσιμη στο $[-α, α]$, τότε ισχύουν οι προτάσεις:
  Α. η f είναι άρτια στο $[-α, α] \Leftrightarrow $ η f΄ είναι περιττή στο $[-α, α]$
  Β. η f είναι περιττή στο $[-α, α] \Leftrightarrow $ η f΄ είναι άρτια στο $[-α, α]$
 \end{proposition}
\end{frame}

\begin{frame}{Πρόταση15}
 \begin{proposition}
  \normalfont Αν η συνάρτηση f είναι παραγωγίσιμη στο διάστημα Δ και εξίσωση f (x) = 0 έχει (ν) διαφορετικές ρίζες στο Δ, τότε η εξίσωση f΄(x) = 0 έχει τουλάχιστον (ν – 1) ρίζες στο Δ.
 \end{proposition}
\end{frame}

\begin{frame}{Πρόταση16}
 \begin{proposition}
  \normalfont Αν η συνάρτηση f : $\Delta \to R$ (Δ: διάστημα) είναι παραγωγίσιμη και ισχύει ${f}'(x)\ne 0$ για κάθε $x\in \Delta $, τότε η f είναι συνάρτηση «1 – 1».
 \end{proposition}
\end{frame}

\begin{frame}{Πρόταση17}
 \begin{proposition}
  \normalfont Αν η συνάρτηση f είναι παραγωγίσιμη στο [α, β] και ${{f}^{'}}(\alpha )<0<{{f}^{'}}(\beta )$, τότε υπάρχει $\xi \in (\alpha ,\,\,\beta )$με ${{f}^{'}}(\xi )=0$.
 \end{proposition}
\end{frame}

\begin{frame}{Πρόταση18}
 \begin{proposition}
  \normalfont Αν f συνεχής στο [α, β] και ${f}'(x)\ne 0$ για κάθε x$\in $(α, β) , τότε η f
  παρουσιάζει ακρότατα μόνο στα α και β.
 \end{proposition}
\end{frame}

\begin{frame}{Πρόταση19}
 \begin{proposition}
  \normalfont Αν η f είναι συνεχής στο Δ και, τότε το ${{x}_{0}}\in \Delta $ δεν μπορεί να είναι συγχρόνως θέση τοπικού ακρότατου και θέση σημείου καμπής.
 \end{proposition}
\end{frame}

\begin{frame}{Πρόταση20}
 \begin{proposition}
  \normalfont Αν η συνάρτηση f είναι κυρτή και έχει ασύμπτωτη την ευθεία ε: y = αx+β, τότε η ${{C}_{f}}$ βρίσκεται πάνω από την ε.
 \end{proposition}
\end{frame}

\begin{frame}{Πρόταση21}
 \begin{proposition}
  \normalfont Έστω η f είναι συνεχής στο [α, β] , παραγωγίσιμη στο (α, β)

  Α. Αν η $f΄$ είναι γνησίως αύξουσα στο (α, β), τότε $\text{f}\left( \frac{\text{ }\!\!\alpha\!\!\text{ + }\!\!\beta\!\!\text{ }}{\text{2}} \right)<\frac{f(\alpha )+f(\beta )}{2}$
  Β. Αν η $f΄$ είναι γνησίως φθίνουσα στο (α, β), τότε $\text{f}\left( \frac{\text{ }\!\!\alpha\!\!\text{ + }\!\!\beta\!\!\text{ }}{\text{2}} \right)\text{}\frac{\text{f( }\!\!\alpha\!\!\text{ )+f( }\!\!\beta\!\!\text{ )}}{\text{2}}$

 \end{proposition}
\end{frame}

\begin{frame}{Πρόταση22}
 \begin{proposition}
  \normalfont Έστω η συνάρτηση f : $[-\alpha ,\alpha ]\to R$είναι συνεχής και
  Α. αν η f είναι άρτια, τότε $\int_{-\alpha }^{\alpha }{f(x)dx}=2\int_{0}^{\alpha }{f(x)dx}$+\=
  Β. αν η f είναι περιττή, τότε $\int_{-\alpha }^{\alpha }{f(x)dx}=0$
 \end{proposition}
\end{frame}

\begin{frame}{Πρόταση23}
 \begin{proposition}
  \normalfont Αν η συνάρτηση f συνεχής στο [α, β] , $f(x)\ge 0$ για κάθε $x\in [\alpha ,\,\beta ]$ και $\int_{\alpha }^{\beta }{f(x)dx}=0$, τότε είναι f(x) = 0 για κάθε $x\in [\alpha ,\,\beta ]$.
 \end{proposition}
\end{frame}

\begin{frame}{Πρόταση24}
 \begin{proposition}
  \normalfont Αν η συνάρτηση f : $\Delta \to R$ , όπου Δ διάστημα, είναι συνεχής στο Δ και
  ισχύει $f(x)\ne 0$ για κάθε $x\in \Delta $, $\int_{\alpha }^{\beta }{f(x)dx}=0$ με $\alpha ,\beta \in \Delta $, τότε είναι α = β.
 \end{proposition}
\end{frame}

\begin{frame}{Πρόταση25}
 \begin{proposition}
  \normalfont Αν η συνάρτηση f : $\Delta \to R$ , όπου Δ διάστημα, είναι συνεχής στο Δ και ισχύει $\alpha ,\beta \in \Delta $, τότε $\int_{\alpha }^{\beta }{f(x)dx}=\int_{\alpha }^{\beta }{f(\alpha +\beta -x)dx}$.
 \end{proposition}
\end{frame}

\begin{frame}[plain,c]
 \begin{center}
  \Huge Σας Ευχαριστούμε...
 \end{center}
\end{frame}

\end{document}
