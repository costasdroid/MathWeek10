\documentclass[greek]{beamer}
%\usepackage{fontspec}
%\newtheorem{definition}{Ορισμός}
\usepackage{amsmath,amsthm}
\usepackage{unicode-math}
\usepackage{xltxtra}
\usepackage{graphicx}
\usetheme{Warsaw}
\usecolortheme{seahorse}
\usepackage{hyperref}
\usepackage{ulem}
\usepackage{xgreek}
\usepackage{pgfpages}
%\setbeameroption{show notes on second screen}
%\setbeameroption{show only notes}

\setsansfont{Times New Roman}
\title[Αντιπαράδειγμα, Αντιθετοαντιστροφή και γράφημα]{Αντιπαράδειγμα, Αντιθετοαντιστροφή και γράφημα στη διδασκαλία της ανάλυσης Γ΄ λυκείου}
\author[Ελευθεριάδης, Λόλας, Ευαγγελόπουλος]{Μ. Ελευθεριάδης\inst{1} \and Κ. Λόλας\inst{2} \and Α. Ευαγγελόπουλος\inst{3}}
\institute[]
{
  \inst{1}%
  32ο ΓΕΛ ΘΕΣ/ΝΙΚΗΣ (ΠΕ03)
  \and
  \inst{2}%
  10ο ΓΕΛ ΘΕΣ/ΝΙΚΗΣ (ΠΕ03)
  \and
  \inst{3}%
  Σχ. Σύμβουλος Μαθηματικών
}
\date{Θεσσαλονίκη, Απρίλιος 2018}

\newtheorem{proposition}[theorem]{Πρόταση}

\begin{document}
\begin{frame}
 \titlepage
\end{frame}

\section{Εισαγωγή}
\begin{frame}{Μαθητές}
 \begin{itemize}
  \item<1-> Συμμετέχουν στη παράδοση
  \item<2-> Λύνουν ασκήσεις
  \item<3-> Ξαναλύνουν ασκήσεις
  \item<4-> Ξαναξαναλύνουν ασκήσεις ...
 \end{itemize}
\end{frame}

\section{Θεωρία}
\begin{frame}{Εργαλεία για τη Θεωρία}
 \begin{itemize}
  \item<1-> Αντιπαράδειγμα
  \item<2-> Αντιθετο-αντιστροφή
  \item<3-> Γραφική Αναπαράσταση
 \end{itemize}
\end{frame}

\subsection{Αντιπαράδειγμα}
\begin{frame}{Αντιπαραδείγματα}
 \begin{center}
  \begin{block}{Χρησιμότητα}
   Ένα παράδειγμα είναι ικανό να αποδείξουμε ότι η πρόταση είναι ψευδής
  \end{block}
 \end{center}
 \begin{itemize}
  \item παράδειγμα μεταστροφής
  \item παράδειγμα γεφύρωσης
 \end{itemize}
\end{frame}

\subsection{Αντιθετοαντιστροφή}
\begin{frame}{Αντιθετο-αντιστροφή}
 \begin{center}
   \begin{block}{Χρησιμότητα}<1->
    Ίσως η απόδειξη είναι ευκολότερη από την αρχική
   \end{block}
   \begin{block}{Χρησιμότητα}<2->
    Μας δίνει την ευκαιρία να το παίξουμε έξυπνοι, ΘΕΟΙ
   \end{block}
 \end{center}
\end{frame}

\section{Παραδείγματα}
\begin{frame}{Πρόταση 1}
 \begin{proposition}
  \normalfont Αν η $f$ συνεχής τότε η $f^{-1}$ δεν είναι πάντα συνεχής.
 \end{proposition}
 \begin{proof}
  Η συνάρτηση $f\left( x \right)=\left\{ \begin{array}{*{35}{l}}
    x,   & x\in \left[ 0,1 \right] \\
    x-1, & x\in \left( 2,3 \right) \\
   \end{array} \right.$ είναι συνεχής ενώ η ${{f}^{-1}}\left( x \right)=\left\{ \begin{array}{*{35}{l}}
    x,   & x\in \left[ 0,1 \right] \\
    x+1, & x\in \left( 1,2 \right) \\
   \end{array} \right.$ δεν είναι συνεχής στο $1$.
 \end{proof}
\end{frame}

\begin{frame}{Πρόταση 2}
 \begin{proposition}
  \normalfont Αν η $f$ παραγωγίσιμη, τότε η $f^{-1}$ δεν είναι πάντα παραγωγίσιμη.
 \end{proposition}
 \begin{proof}
  Η συνάρτηση $f\left( x \right)={{x}^{3}}$ είναι «1-1» και παραγωγίσιμη στο $\mathbb{R}$. Όμως η ${{f}^{-1}}\left( x \right)=\left\{ \begin{array}{*{35}{l}}
    \sqrt[3]{x},   & x\ge 0 \\
    -\sqrt[3]{-x}, & x<0    \\
   \end{array} \right.$ δεν είναι παραγωγίσιμη στο 0.
 \end{proof}
\end{frame}

\begin{frame}{Πρόταση 3}
 \begin{proposition}
  \normalfont Αν υπάρχει το όριο $\underset{x\to {{x}_{0}}}{\mathop{\lim }}\,\left[ f\left( x \right)+g\left( x \right) \right]$, τότε δεν υπάρχουν πάντα τα όρια $\underset{x\to {{x}_{0}}}{\mathop{\lim }}\,f\left( x \right)$ και $\underset{x\to {{x}_{0}}}{\mathop{\lim }}\,g\left( x \right)$.
 \end{proposition}
 \begin{proof}
  Για $f\left( x \right)=\frac{1}{x}$ και $g\left( x \right)=-\frac{1}{x}$ τότε $\underset{x\to 0}{\mathop{\lim }}\,\left[ f\left( x \right)+g\left( x \right) \right]=0$ και $\underset{x\to 0}{\mathop{\lim }}\,f\left( x \right)=+\infty $.
 \end{proof}
\end{frame}

\begin{frame}{Πρόταση 4}
 \begin{proposition}
  \normalfont Αν $\underset{x\to +\infty }{\mathop{\lim }}\,f\left( x \right)=+\infty $, τότε η $f$ δεν είναι πάντα γνησίως αύξουσα στο $+\infty $.
 \end{proposition}
 \begin{proof}
  Η συνάρτηση $f\left( x \right)=\eta \mu x+\frac{x}{2}$ έχει $\underset{x\to +\infty }{\mathop{\lim }}\,f\left( x \right)=+\infty $, όμως δεν είναι γνησίως αύξουσα στο $+\infty $.
 \end{proof}
\end{frame}

\begin{frame}{Πρόταση 5}
 \begin{proposition}
  \normalfont Αν η $f$ είναι συνεχής στο ${{x}_{0}}$, τότε η $f$ δεν είναι πάντα παραγωγίσιμη στο ${{x}_{0}}$.
 \end{proposition}
 \begin{proof}
  Η συνάρτηση $f\left( x \right)=\left\{ \begin{array}{*{35}{l}}
    \ln x+1,      & x>1    \\
    -{{x}^{2}}+2, & x\le 1 \\
   \end{array} \right.$ είναι συνεχής στο $1$, όμως δεν είναι παραγωγίσιμη στο $1$.
 \end{proof}
\end{frame}

\begin{frame}{Πρόταση 6}
 \begin{proposition}
  \normalfont Αν η $f$είναι συνεχής και $f(x)\ne 0$ για κάθε $x\in (\alpha ,\beta )\cup (\beta ,\gamma )$, τότε δεν έχει πάντα σταθερό πρόσημο στο  $(\alpha ,\beta )\cup (\beta ,\gamma )$
 \end{proposition}
 \begin{proof}
  Η συνάρτηση $f\left( x \right)=\left\{ \begin{array}{*{35}{l}}
    {{x}^{2}}+1, & x>1 \\
    -2,          & x<1 \\
   \end{array} \right.$είναι συνεχής όμως δεν έχει σταθερό πρόσημο στο $(-\infty ,1)\cup (1,+\infty )$.

 \end{proof}
\end{frame}

\begin{frame}{Πρόταση 7}
 \begin{proposition}
  \normalfont Αν η ${{C}_{f}}$ δέχεται εφαπτομένη $\varepsilon $ στο ${{x}_{0}}$, τότε η $\varepsilon $ δεν έχει πάντα μόνο ένα κοινό σημείο με την ${{C}_{f}}$ (άσκηση Γ10, σελ. 174-175).
 \end{proposition}
 \begin{proof}
  Η συνάρτηση $f\left( x \right)=\left\{ \begin{array}{*{35}{l}}
    {{x}^{2}}\eta \mu \frac{1}{x}, & x\ne 0 \\
    0,                             & x=0    \\
   \end{array} \right.$ είναι παραγωγίσιμη στο $0$ και έχει εφαπτόμενη την $\varepsilon :y=0$ (άξονας $x'x$) που τέμνει την ${{C}_{f}}$ σε άπειρα σημεία (${{x}_{k}}=\frac{1}{k\pi }$, $k\in {{\mathbb{Z}}^{*}}$) παρόλο που εφάπτεται της ${{C}_{f}}$.
 \end{proof}
\end{frame}

\begin{frame}{Πρόταση 8}
 \begin{proposition}
  \normalfont Αν η $f+g$ είναι παραγωγίσιμη στο${{x}_{0}}$, τότε οι $f$, $g$ δεν είναι πάντα παραγωγίσιμες στο ${{x}_{0}}$.
 \end{proposition}
 \begin{proof}
  H συνάρτηση $\left( f+g \right)\left( x \right)=x+1$ είναι παραγωγίσιμη στο $0$, ενώ οι συναρτήσεις $f\left( x \right)=x+\left| x \right|$ και $g\left( x \right)=1-\left| x \right|$ δεν είναι παραγωγίσιμες στο $0$.
 \end{proof}
\end{frame}

\begin{frame}{Πρόταση 9}
 \begin{proposition}
  \normalfont Aν συνάρτηση $f$ είναι ορισμένη και συνεχής στο $\left[ \alpha ,\beta  \right)$ δεν παρουσιάζει πάντα ακρότατο στο $\alpha $.
 \end{proposition}
 \begin{proof}
  H συνάρτηση ${{f}^{-1}}\left( x \right)=\left\{ \begin{array}{*{35}{l}}
    {{x}^{2}}\eta \mu \frac{1}{x}, & x\ne 0 \\      0, & x=0  \\
   \end{array} \right.$ είναι συνεχής ως γινόμενο συνεχών και σύνθετη συνεχών και στο $0$ είναι συνεχής, $\underset{x\to 0}{\mathop{\lim }}\,\left( {{x}^{2}}\eta \mu \frac{1}{x} \right)=0=f\left( 0 \right)$. Το $f\left( 0 \right)$ δεν είναι τοπικό ακρότατο, γιατί όσο κοντά στο $0$, σε διάστημα της μορφής
  $\left[ 0,x \right]$ η $f\left( x \right)$ «αλλάζει» πρόσημο.
 \end{proof}
\end{frame}

\begin{frame}{Πρόταση 10}
 \begin{proposition}
  \normalfont Aν η συνάρτηση $f$ είναι ορισμένη στο $\left[ \alpha ,\beta  \right]$, δεν είναι συνεχής στο ${{x}_{0}}\in \left( \alpha ,\beta  \right)$ και αλλάζει το πρόσημο της $f'$ εκατέρωθεν του ${{x}_{0}}$, τότε δεν παρουσιάζει πάντα ακρότατο στο ${{x}_{0}}$.
 \end{proposition}
 \begin{proof}
  H συνάρτηση $f\left( x \right)=\left\{ \begin{array}{*{35}{l}}
    \sqrt{-x},   & x<0    \\
    {{x}^{2}}+1, & x\ge 0 \\
   \end{array} \right.$ δεν είναι συνεχής στο $0$. Είναι γνησίως φθίνουσα στο $\left( -\infty ,0 \right)$ και γνησίως αύξουσα στο $\left( 0,+\infty  \right)$, όμως δε παρουσιάζει τ. ακρότατο στο $0$.
 \end{proof}
\end{frame}

\begin{frame}{Πρόταση 11}
 \begin{proposition}
  \normalfont Αν η ${{C}_{f}}$ μιας συνάρτησης $f$, αλλάζει κυρτότητα εκατέρωθεν του ${{x}_{0}}$, τότε δεν παρουσιάζει πάντα στο ${{x}_{0}}$ σημείο καμπής.
 \end{proposition}
 \begin{proof}
  Έστω η συνάρτηση $f\left( x \right)=\left\{ \begin{array}{*{35}{l}}
    \sqrt{-x}, & x<0    \\
    {{x}^{2}}, & x\ge 0 \\
   \end{array} \right.$. Στο ${{x}_{0}}=0$ η κυρτότητα αλλάζει είδος, αφού $f''\left( x \right)=\left\{ \begin{array}{*{35}{l}}
    \frac{1}{4x\sqrt{-x}}, & x<0    \\
    2,                     & x\ge 0 \\
   \end{array} \right.$ (δεν υπάρχει η $f''\left( 0 \right)$, αφού δεν υπάρχει και η $f'\left( 0 \right)$) και $f''\left( 0 \right)<0$ για $x<0$ και $f''\left( 0 \right)>0$ για $x>0$. Στο ${{x}_{0}}=0$ δεν υπάρχει εφαπτομένη και επομένως το ${{x}_{0}}=0$ δεν είναι σημείο καμπής.
 \end{proof}
\end{frame}

\begin{frame}{Πρόταση 12}
 \begin{proposition}
  \normalfont Αν η ${{C}_{f}}$ μιας συνάρτησης $f$, έχει ασύμπτωτη την ευθεία $\varepsilon $, τότε η ${{C}_{f}}$ μπορεί να τέμνει την $\varepsilon $.
 \end{proposition}
 \begin{proof}
  Η συνάρτηση $f\left( x \right)=\left\{ \begin{array}{*{35}{l}}
    \frac{\eta \mu x}{x}, & x\ne 0 \\
    1,                    & x=0    \\
   \end{array} \right.$ είναι συνεχής και $\underset{x\to +\infty }{\mathop{\lim }}\,\frac{\eta \mu x}{x}=0$ και $\underset{x\to -\infty }{\mathop{\lim }}\,\frac{\eta \mu x}{x}=0$
  άρα η ευθεία $y=0$ είναι οριζόντια ασύμπτωτη της $f$ (δηλαδή ο άξονας $x'x$). Όμως η εξίσωση $f\left( x \right)=0$ έχει άπειρες λύσεις στο $\mathbb{R}$, άρα ο άξονας $x'x$ και η ${{C}_{f}}$ έχουν άπειρα κοινά σημεία ($x=k\pi $, $k\in \mathbb{Z}$).
 \end{proof}
\end{frame}

\begin{frame}{Πρόταση 13}
 \begin{proposition}
  \normalfont Αντίστροφο Θ. Rolle. Αν η $f$ είναι συνεχής στο $\left[ \alpha ,\beta  \right]$, $f\left( \alpha  \right)=f\left( \beta  \right)$ και υπάρχει ${{x}_{0}}\in \left( \alpha ,\beta  \right)$ τέτοιο ώστε $f'\left( {{x}_{0}} \right)=0$, τότε η $f$ δεν είναι πάντα παραγωγίσιμη στο $\left( \alpha ,\beta  \right)$.
 \end{proposition}
 \begin{proof}
  Η συνάρτηση $f\left( x \right)=\left| {{x}^{2}}-4x \right|$ έχει $f\left( -1 \right)=f\left( 5 \right)=5$ και $f'\left( 2 \right)=0$, αλλά δεν είναι παραγωγίσιμη στο $0$.
 \end{proof}
\end{frame}

\begin{frame}{Πρόταση 14}
 \begin{proposition}
  \normalfont Αν η $f$ είναι παραγωγίσιμη στο $\left( \alpha ,\beta  \right)$, $f\left( \alpha  \right)=f\left( \beta  \right)$ και υπάρχει ${{x}_{0}}\in \left( \alpha ,\beta  \right)$ τέτοιο ώστε $f'\left( x_0 \right)=0$, τότε η $f$ δεν είναι πάντα συνεχής στο $\left[ \alpha ,\beta  \right]$.
 \end{proposition}
\end{frame}

\begin{frame}{Πρόταση 15}
 \begin{proposition}
  \normalfont Αν η $f$ είναι συνεχής στο $\left[ \alpha ,\beta  \right]$, παραγωγίσιμη στο $\left( \alpha ,\beta  \right)$ και υπάρχει $x_0\in \left( \alpha ,\beta  \right)$ τέτοιο ώστε $f'\left( x_0 \right)=0$, τότε δεν είναι πάντα $f\left( \alpha  \right)=f\left( \beta  \right)$.
 \end{proposition}
 \begin{proof}
  Η συνάρτηση $f\left( x \right)={{x}^{2}}$, $x\in \left[ -1,2 \right]$ είναι παραγωγίσιμη στο $\left( -1,2 \right)$, $f'\left( 0 \right)=0$, αλλά $f\left( -1 \right)\ne f\left( 2 \right)$.
 \end{proof}
\end{frame}

\begin{frame}{Πρόταση 16}
 \begin{proposition}
  \normalfont Αντίστροφο ΘΜΤ. Αν η $f$ είναι συνεχής στο $\left[ \alpha ,\beta  \right]$ και υπάρχει ${{x}_{0}}\in \left( \alpha ,\beta  \right)$ τέτοιο ώστε $f'\left( x_0 \right)=\frac{f\left( \beta  \right)-f\left( \alpha  \right)}{\beta -\alpha }$, τότε η $f$ δεν είναι πάντα παραγωγίσιμη στο $\left( \alpha ,\beta  \right)$.
 \end{proposition}
\end{frame}

\begin{frame}{Πρόταση 17}
 \begin{proposition}
  \normalfont Αν η $f$ είναι παραγωγίσιμη στο $\left( \alpha ,\beta  \right)$ και υπάρχει ${{x}_{0}}\in \left( \alpha ,\beta  \right)$ τέτοιο ώστε $f'\left( x_0 \right)=\frac{f\left( \beta  \right)-f\left( \alpha  \right)}{\beta -\alpha }$, τότε η $f$ δεν είναι πάντα συνεχής στο $\left[ \alpha ,\beta  \right]$.
 \end{proposition}
\end{frame}

\begin{frame}{Πρόταση 18}
 \begin{proposition}
  \normalfont Αντίστροφο Μονοτονίας. Αν η $f$ είναι συνεχής στο $\left[ \alpha ,\beta  \right]$, παραγωγίσιμη στο $\left( \alpha ,\beta  \right)$ και γνησίως αύξουσα στο $\left[ \alpha ,\beta  \right]$, τότε δεν είναι πάντα $f'\left( x  \right)>0$, για κάθε $x\in \left( \alpha ,\beta  \right)$.
 \end{proposition}
\end{frame}

\begin{frame}{Πρόταση 19}
 \begin{proposition}
  \normalfont Αν $f'\left( x  \right)>0$ για κάθε $x\in \left( \alpha ,\beta  \right)$ και $f$ γνησίως αύξουσα στο $\left[ \alpha ,\beta  \right]$, τότε δεν είναι πάντα η $f$ συνεχής στο $\left[ \alpha ,\beta  \right]$.
 \end{proposition}
\end{frame}

\begin{frame}{Πρόταση 20}
 \begin{proposition}
  \normalfont Αντίστροφο Θ. Fermat. Αν η $f$ είναι παραγωγίσιμη ${{x}_{0}}\in \left( \alpha ,\beta  \right)$, $f'\left( {{x}_{0}} \right)=0$, τότε η $f$ δεν είναι πάντα θέση τ. ακρότατου το ${{x}_{0}}$.
 \end{proposition}
\end{frame}

\begin{frame}{Πρόταση 21}
 \begin{proposition}
  \normalfont Αν η f έχει θέση τοπ. ακρότατου το ${{x}_{0}}$ και $f'\left( {{x}_{0}} \right)=0$, τότε το ${{x}_{0}}$ δεν είναι πάντα εσωτερικό σημείο διαστήματος.
 \end{proposition}
\end{frame}

\begin{frame}{Πρόταση 22}
 \begin{proposition}
  \normalfont Αν το $x\underset{x\to xo}{\mathop{\lim }}\,\frac{f\left( x \right)}{g\left( x \right)}=\left( \frac{0}{0} \right)$ ή $\left( \frac{\pm \infty }{\pm \infty } \right)$ και υπάρχει τότε δεν υπάρχει πάντα το $\underset{x\to xo}{\mathop{\lim }}\,\frac{f'\left( x \right)}{g'\left( x \right)}$.
 \end{proposition}
 \begin{proof}
  Για $f\left( x \right)={{x}^{2}}\eta \mu \frac{1}{x}$ και $g\left( x \right)=\varepsilon \varphi x$ έχουμε $$\underset{x\to 0}{\mathop{\lim }}\,\frac{{{x}^{2}}\eta \mu \frac{1}{x}}{\varepsilon \varphi x}=\left( \frac{0}{0} \right)=\underset{x\to 0}{\mathop{\lim }}\,\frac{x\eta \mu \frac{1}{x}}{\frac{\eta \mu x}{x}\frac{1}{\sigma \upsilon \nu x}}=0$$
    αλλά
  $$\underset{x\to 0}{\mathop{\lim }}\,\frac{\left( {{x}^{2}}\eta \mu \frac{1}{x} \right)'}{\left( \varepsilon \varphi x \right)'}=\underset{x\to 0}{\mathop{\lim }}\,\frac{2x\text{ }\!\!\eta\!\!\text{  }\!\!\mu\!\!\text{ }\frac{1}{x}-\text{ }\!\!\sigma\!\!\text{  }\!\!\upsilon\!\!\text{  }\!\!\nu\!\!\text{ }\frac{1}{x}}{\frac{1}{\text{ }\!\!\sigma\!\!\text{  }\!\!\upsilon\!\!\text{ }{{\text{ }\!\!\nu\!\!\text{ }}^{2}}x}}=\underset{x\to 0}{\mathop{\lim }}\,\sigma \upsilon {{\nu }^{2}}x\left( 2x\text{ }\!\!\eta\!\!\text{  }\!\!\mu\!\!\text{ }\frac{1}{x}-\text{ }\!\!\sigma\!\!\text{  }\!\!\upsilon\!\!\text{  }\!\!\nu\!\!\text{ }\frac{1}{x} \right)$$
 \end{proof}
\end{frame}

\begin{frame}{Πρόταση 23}
 \begin{proposition}
  \normalfont Αντίστροφο Σ.Κ. Αν η $f$ είναι 2 φορές παραγωγίσιμη  και  $f'\left( {{x}_{0}} \right)=0$, τότε η $f$ δεν είναι πάντα θέση σημείου καμπής.
 \end{proposition}
 \begin{proof}
  Η $f\left( x \right)={{x}^{4}}$, έχει $f'\left( {{x}_{0}} \right)=0$, αλλά το $0$ δεν είναι σημείο καμπής.
 \end{proof}
\end{frame}

\begin{frame}{Πρόταση 24}
 \begin{proposition}
  \normalfont Αντίστροφο Κυρτότητας. Αν η $f$ είναι συνεχής στο $\left[ \alpha ,\beta  \right]$, 2 φορές παραγωγίσιμη στο $\left( \alpha ,\beta  \right)$ και κυρτή στο $\left[ \alpha ,\beta  \right]$, τότε δεν είναι $f''\left( x  \right)>0$, για κάθε $x\in \left( \alpha ,\beta  \right)$.
 \end{proposition}
 \begin{proof}
  Η $f\left( x \right)={{\left( x-2 \right)}^{4}}+x$ είναι κυρτή αλλά έχει $f''\left( 2 \right)=0$.
 \end{proof}
\end{frame}

\begin{frame}{Πρόταση 25}
 \begin{proposition}
  \normalfont Πρόταση 25.	Αν ${f}'(x)\ne 0$ για κάθε $x\in (\alpha ,\beta )\cup (\beta ,\gamma )$, τότε δεν είναι πάντα γνησίως αύξουσα στο  $(\alpha ,\beta )\cup (\beta ,\gamma )$
 \end{proposition}
 \begin{proof}
  Η συνάρτηση $f\left( x \right)=-\frac{1}{x}$είναι${f}'\left( x \right)=\frac{1}{{{x}^{2}}}>0$
  για $x\in (-\infty ,0)\cup (0,+\infty )$ όμως $f\left( -1 \right)=1>f(1)=-1$
 \end{proof}
\end{frame}

\begin{frame}{Πρόταση 26}
 \begin{proposition}
  \normalfont Αν η $f$ είναι παραγωγίσιμη, τότε η $f'$ δεν είναι πάντα συνεχής.
 \end{proposition}
 \begin{proof}
  Η συνάρτηση $f\left( x \right)=\left\{ \begin{array}{*{35}{l}}
    {{x}^{2}}\eta \mu \frac{1}{x}, & x\ne 0 \\
    0,                             & x=0    \\
   \end{array} \right.$ είναι συνεχής στο $\mathbb{R}$, με παράγωγο $f'\left( x \right)=\left\{ \begin{array}{*{35}{l}}
    2x\eta \mu \frac{1}{x}-\sigma \upsilon \nu \frac{1}{x}, & x\ne 0 \\
    0,                                                      & x=0    \\
   \end{array} \right.$ η οποία δεν είναι συνεχής στο $0$.
 \end{proof}
\end{frame}

\begin{frame}{Πρόταση 27}
 \begin{proposition}
  \normalfont Αν η $f$ δεν είναι αντιστρέψιμη τότε δεν είναι «1-1».
 \end{proposition}
 \begin{proof}
  Αν ήταν «1-1» τότε θα ήταν αντιστρέψιμη.
 \end{proof}
\end{frame}

\begin{frame}{Πρόταση 28}
 \begin{proposition}
  \normalfont Αν δεν υπάρχει το όριο $\underset{x\to {{x}_{0}}}{\mathop{\lim }}\,\left[ f\left( x \right)+g\left( x \right) \right]$, τότε δεν υπάρχει το $\underset{x\to {{x}_{0}}}{\mathop{\lim }}\,f\left( x \right)$ ή το $\underset{x\to {{x}_{0}}}{\mathop{\lim }}\,g\left( x \right)$.
 \end{proposition}
 \begin{proof}
  Αν υπήρχαν και τα δύο, τότε τα υπήρχε και το $\underset{x\to {{x}_{0}}}{\mathop{\lim }}\,\left[ f\left( x \right)+g\left( x \right) \right]$, ΑΤΟΠΟ.
 \end{proof}
\end{frame}

\begin{frame}{Πρόταση 29}
 \begin{proposition}
  \normalfont Αν η $g\circ f$ δεν είναι συνεχής στο ${{x}_{0}}$, τότε η $f$ δεν είναι συνεχής στο ${{x}_{0}}$ ή η $g$ δεν είναι συνεχής στο $f\left( {{x}_{0}} \right)$.
 \end{proposition}
 \begin{proof}
  Αν η $f$ ήταν συνεχής στο ${{x}_{0}}$ και η $g$ ήταν συνεχής στο $f\left( {{x}_{0}} \right)$ τότε θα ήταν συνεχής και η $g\circ f$, ΑΤΟΠΟ.
 \end{proof}
\end{frame}

\begin{frame}{Πρόταση 30}
 \begin{proposition}
  \normalfont Αν η $f$ δεν έχει ρίζα στο $\left( \alpha ,\beta  \right)$, τότε η $f$ δεν είναι συνεχής στο $\left[ \alpha ,\beta  \right]$ ή  $f\left( \alpha  \right)\cdot f\left( \beta  \right)\ge 0$
 \end{proposition}
 \begin{proof}
  Αν η $f$ ήταν συνεχής στο $\left[ \alpha ,\beta  \right]$ και $f\left( \alpha  \right)\cdot f\left( \beta  \right)<0$ τότε η $f$ έχει ρίζα στο $\left( \alpha ,\beta  \right)$, ΑΤΟΠΟ.
 \end{proof}
\end{frame}

\begin{frame}{Πρόταση 31}
 \begin{proposition}
  \normalfont Αν η $g\circ f$ δεν είναι παραγωγίσιμη στο ${{x}_{0}}$, τότε η $f$ δεν είναι πραγωγίσιμη στο ${{x}_{0}}$ ή η $g$ δεν είναι παραγωγίσιμη στο $f\left( {{x}_{0}} \right)$.
 \end{proposition}
 \begin{proof}
  Αν η $f$ ήταν παραγωγίσιμη στο ${{x}_{0}}$ και η $g$ ήταν παραγωγίσιμη στο $f\left( {{x}_{0}} \right)$ τότε θα ήταν παραγωγίσιμη και η $g\circ f$ στο ${{x}_{0}}$, ΑΤΟΠΟ.
 \end{proof}
\end{frame}

\begin{frame}{Πρόταση 32}
 \begin{proposition}
  \normalfont Αν η $f$ δεν είναι συνεχής στο ${{x}_{0}}$, τότε η $f$ δεν είναι πραγωγίσιμη στο ${{x}_{0}}$.
 \end{proposition}
 \begin{proof}
  Αν η $f$ ήταν παραγωγίσιμη στο ${{x}_{0}}$ τότε θα ήταν και συνεχής στο ${{x}_{0}}$, ΑΤΟΠΟ.
 \end{proof}
\end{frame}

\begin{frame}[plain,c]
 \begin{center}
  \Huge Σας Ευχαριστούμε...
 \end{center}
\end{frame}

\end{document}
