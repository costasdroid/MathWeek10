\documentclass[greek]{beamer}
%\usepackage{fontspec}
%\newtheorem{definition}{Ορισμός}
\usepackage{amsmath,amsthm}
\usepackage{unicode-math}
\usepackage{xltxtra}
\usepackage{graphicx}
\usetheme{Warsaw}
\usecolortheme{seahorse}
\usepackage{hyperref}
\usepackage{ulem}
\usepackage{xgreek}
\usepackage{pgfpages}
%\setbeameroption{show notes on second screen}
%\setbeameroption{show only notes}

\setsansfont{Times New Roman}
\title[Αντιπαράδειγμα, Αντιθετοαντιστροφή και γράφημα]{Αντιπαράδειγμα, Αντιθετοαντιστροφή και γράφημα στη διδασκαλία της ανάλυσης Γ΄ λυκείου}
\author[Ελευθεριάδης, Λόλας, Ευαγγελόπουλος]{Μ. Ελευθεριάδης\inst{1} \and Κ. Λόλας\inst{2} \and Α. Ευαγγελόπουλος\inst{3}}
\institute[]
{
  \inst{1}%
  32ο ΓΕΛ ΘΕΣ/ΝΙΚΗΣ (ΠΕ03)
  \and
  \inst{2}%
  10ο ΓΕΛ ΘΕΣ/ΝΙΚΗΣ (ΠΕ03)
  \and
  \inst{3}%
  Σχ. Σύμβουλος Μαθηματικών
}
\date{Θεσσαλονίκη, Απρίλιος 2018}

\newtheorem{proposition}[theorem]{Πρόταση}

\begin{document}
\begin{frame}
 \titlepage
\end{frame}

\section{Εισαγωγή}
\begin{frame}{Μαθητές}
 \begin{itemize}
  \item<1-> Συμμετέχουν στη παράδοση
  \item<2-> Λύνουν ασκήσεις
  \item<3-> Ξαναλύνουν ασκήσεις
  \item<4-> Ξαναξαναλύνουν ασκήσεις ...
 \end{itemize}
\end{frame}

\section{Θεωρία}
\begin{frame}{Εργαλεία για τη Θεωρία}
 \begin{itemize}
  \item<1-> Αντιπαράδειγμα
  \item<2-> Αντιθετο-αντιστροφή
  \item<3-> Γραφική Αναπαράσταση
 \end{itemize}
\end{frame}

\subsection{Αντιπαράδειγμα}
\begin{frame}{Αντιπαραδείγματα}
 \begin{center}
  \begin{block}{Χρησιμότητα}
   Ένα παράδειγμα είναι ικανό να αποδείξουμε ότι η πρόταση είναι ψευδής
  \end{block}
 \end{center}
 \begin{itemize}
  \item παράδειγμα μεταστροφής
  \item παράδειγμα γεφύρωσης
 \end{itemize}
\end{frame}

\subsection{Αντιθετοαντιστροφή}
\begin{frame}{Αντιθετο-αντιστροφή}
 \begin{center}
  \begin{block}{Χρησιμότητα}
   Ίσως η απόδειξη είναι ευκολότερη από την αρχική
  \end{block}
 \end{center}
\end{frame}

\section{Παραδείγματα}
\begin{frame}{Πρόταση}
 \begin{proposition}
  \normalfont Αν η $f$ συνεχής τότε η $f^{-1}$ δεν είναι πάντα συνεχής.
 \end{proposition}
 \begin{proof}
  Η συνάρτηση $f\left( x \right)=\left\{ \begin{array}{*{35}{l}}
    x,   & x\in \left[ 0,1 \right] \\
    x-1, & x\in \left( 2,3 \right) \\
   \end{array} \right.$ είναι συνεχής ενώ η ${{f}^{-1}}\left( x \right)=\left\{ \begin{array}{*{35}{l}}
    x,   & x\in \left[ 0,1 \right] \\
    x+1, & x\in \left( 1,2 \right) \\
   \end{array} \right.$ δεν είναι συνεχής στο $1$.
 \end{proof}
\end{frame}

\begin{frame}{Πρόταση}
 \begin{proposition}
  \normalfont Αν η $f$ παραγωγίσιμη, τότε η $f^{-1}$ δεν είναι πάντα παραγωγίσιμη.
 \end{proposition}
 \begin{proof}
  Η συνάρτηση $f\left( x \right)={{x}^{3}}$ είναι «1-1» και παραγωγίσιμη στο $\mathbb{R}$. Όμως η ${{f}^{-1}}\left( x \right)=\left\{ \begin{array}{*{35}{l}}
    \sqrt[3]{x},   & x\ge 0 \\
    -\sqrt[3]{-x}, & x<0    \\
   \end{array} \right.$ δεν είναι παραγωγίσιμη στο 0.
 \end{proof}
\end{frame}

\begin{frame}{Πρόταση}
 \begin{proposition}
  \normalfont Αν υπάρχει το όριο $\underset{x\to {{x}_{0}}}{\mathop{\lim }}\,\left[ f\left( x \right)+g\left( x \right) \right]$, τότε δεν υπάρχουν πάντα τα όρια $\underset{x\to {{x}_{0}}}{\mathop{\lim }}\,f\left( x \right)$ και $\underset{x\to {{x}_{0}}}{\mathop{\lim }}\,g\left( x \right)$.
 \end{proposition}
 \begin{proof}
  Για $f\left( x \right)=\frac{1}{{{x}^{2}}}$ και $g\left( x \right)=-\frac{1}{{{x}^{2}}}$ τότε $\underset{x\to 0}{\mathop{\lim }}\,\left[ f\left( x \right)+g\left( x \right) \right]=0$ και $\underset{x\to 0}{\mathop{\lim }}\,f\left( x \right)=+\infty $.
 \end{proof}
\end{frame}

\begin{frame}{Πρόταση}
 \begin{proposition}
  \normalfont Αν $\underset{x\to +\infty }{\mathop{\lim }}\,f\left( x \right)=+\infty $, τότε η $f$ δεν είναι πάντα γνησίως αύξουσα στο $+\infty $.
 \end{proposition}
 \begin{proof}
  Η συνάρτηση $f\left( x \right)=\eta \mu x+x$ έχει $\underset{x\to +\infty }{\mathop{\lim }}\,f\left( x \right)=+\infty $, όμως δεν είναι γνησίως αύξουσα στο $+\infty $.
 \end{proof}
\end{frame}

\begin{frame}{Πρόταση}
 \begin{proposition}
  \normalfont Αν η $f$ είναι συνεχής στο ${{x}_{0}}$, τότε η $f$ δεν είναι πάντα παραγωγίσιμη στο ${{x}_{0}}$.
 \end{proposition}
 \begin{proof}
   Η συνάρτηση $f\left( x \right)=\left\{ \begin{array}{*{35}{l}}
      \ln x+1, & x>1  \\
      -{{x}^{2}}+2, & x\le 1  \\
   \end{array} \right.$ είναι συνεχής στο $1$, όμως δεν είναι παραγωγίσιμη στο $1$.
 \end{proof}
\end{frame}

\begin{frame}{Πρόταση}
 \begin{proposition}
  \normalfont Αν η ${{C}_{f}}$ δέχεται εφαπτομένη $\varepsilon $ στο ${{x}_{0}}$, τότε η $\varepsilon $ δεν έχει πάντα μόνο ένα κοινό σημείο με την ${{C}_{f}}$ (άσκηση Γ10, σελ. 174-175).
 \end{proposition}
 \begin{proof}
   Η συνάρτηση $f\left( x \right)=\left\{ \begin{array}{*{35}{l}}
      {{x}^{2}}\eta \mu \frac{1}{x}, & x\ne 0  \\
      0, & x=0  \\
   \end{array} \right.$ είναι παραγωγίσιμη στο $0$ και έχει εφαπτόμενη την $\varepsilon :y=0$ (άξονας $x'x$) που τέμνει την ${{C}_{f}}$ σε άπειρα σημεία (${{x}_{k}}=\frac{1}{k\pi }$, $k\in {{\mathbb{Z}}^{*}}$) παρόλο που εφάπτεται της ${{C}_{f}}$.
 \end{proof}
\end{frame}

\begin{frame}{Πρόταση}
 \begin{proposition}
  \normalfont Αν η $f+g$ είναι παραγωγίσιμη στο${{x}_{0}}$, τότε οι $f$, $g$ δεν είναι πάντα παραγωγίσιμες στο ${{x}_{0}}$.
 \end{proposition}
 \begin{proof}
  H συνάρτηση $\left( f+g \right)\left( x \right)=x+1$ είναι παραγωγίσιμη στο $0$, ενώ οι συναρτήσεις $f\left( x \right)=x+\left| x \right|$ και $g\left( x \right)=1-\left| x \right|$ δεν είναι παραγωγίσιμες στο $0$.
 \end{proof}
\end{frame}

\begin{frame}[plain,c]
 \begin{center}
  \Huge Σας Ευχαριστούμε...
 \end{center}
\end{frame}

\end{document}
